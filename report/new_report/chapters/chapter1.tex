\chapter{Εισαγωγή}
\label{chapter:intro}

\setlength{\parskip}{1em}
Τα τελευταία χρόνια η τεχνολογική πρόοδος έχει οδηγήσει τον κόσμο σε αυξημένη αλληλεπίδραση μεταξύ του φυσικού και ψηφιακού καθώς ένας μεγάλος αριθμός πληροφοριών συλλέγεται, επεξεργάζεται στο παρασκήνιο και αναφέρεται σε κατάλληλες στιγμές. Οι συσκευές αυτές συνδυάζουν τεχνολογίες λογισμικού και υλικού προκειμένου να συνδεθούν με άλλες συσκευές ή αισθητήρες και να αυτοματοποιήσουν διάφορες εργασίες.

Οι έξυπνες συσκευές για το σπίτι αποτελούν ένα παράδειγμα συσκευών όπου η ευρεία τους χρήση μπορεί να βελτιώσει την καθημερινότητά μας, να βοηθήσει στις καθημερινές εργασίες και ταυτόχρονα να προσφέρουν διασκέδαση. Ένας έξυπνος θερμοστάτης, για παράδειγμα, μπορεί να ρυθμίζει αυτόματα την θερμοκρασία του σπιτιού διατηρώντας έτσι μια επιθυμητή τιμή και εξοικονομώντας ενέργεια.

Μια έξυπνη συσκευή για το σπίτι η οποία έχει αρκετές πιθανές εφαρμογές είναι ο έξυπνος καθρέφτης. Η συσκευή αυτή επαυξάνει την ανακλαστική επιφάνεια με υλικό και λογισμικό που επιτρέπουν την συλλογή και εμφάνιση πληροφοριών στον χρήστη, 



\section{Περιγραφή του Προβλήματος}
\label{section:problem_description}

Οι περισσότερες εφαρμογές που έχουν αναπτυχθεί πάνω στον έξυπνο καθρέφτη αυτή τη στιγμή αφορούν κυρίως στην ενημέρωση (news and social media feeds), στον συγχρονισμό ηλεκτρονικού ημερολογίου για λήψη υπενθυμίσεων και στην προβολή του καιρού αποτελώντας αρωγό σε καθημερινές δραστηριότητες. Όμως, οι απαιτήσεις χρηστών των παραπάνω εφαρμογών ικανοποιούνται ευκολότερα και αποτελεσματικότερα από άλλες συσκευές, όπως smartphone, καθιστώντας τον έξυπνο καθρέφτη μη ελκυστικό εμπορικό προϊόν. 

Βασική αιτία των παραπάνω αποτελεί το εγγενές πρόβλημα αλληλεπίδρασης με τον καθρέφτη, δηλαδή η έλλειψη αποδοτικών συσκευών εισόδου. Η επιλογή που φαντάζει λογικότερη ως είσοδο αποτελεί η φωνή του χρήστη δεδομένου ότι η επαφή του με τον καθρέφτη γίνεται από απόσταση. Όμως, στην ανίχνευση φωνής ελλοχεύουν δύο κύρια προβλήματα. Αρχικά, η αβεβαιότητα των συνθηκών του περιβάλλοντος στο οποίο τοποθετείται ο καθρέφτης δυσχεραίνει την ικανότητα του συστήματος να ανιχνεύει καθαρά τον ήχο. Επιπλέον, οι σημερινές τεχνολογίες αναγνώρισης φωνής και κατανόησης πρόθεσης του χρήστη επιδέχονται βελτίωση.

Επίσης, έως και σήμερα δεν έχει υλοποιηθεί κάποιο λειτουργικό σύστημα, το οποίο να χρησιμοποιείται καθολικά, κανονικοποιώντας έτσι τον τρόπο ανάπτυξης εφαρμογών πάνω στον έξυπνο καθρέφτη. Επομένως, οι λειτουργίες του κάθε συστήματος καθορίζονται από τον εκάστοτε κατασκευαστή μειώνοντας με αυτόν τον τρόπο την επεκτασιμότητα του έξυπνου καθρέφτη.

Παρόλα αυτά, ο έξυπνος καθρέφτης αποτελεί πρόσφορο έδαφος για την ανάπτυξη εφαρμογών που επαυξάνουν την βασική του λειτουργία, την προβολή του ειδώλου. Συνεχώς ο άνθρωπος αξιοποιεί έναν καθρέφτη προκειμένου να πάρει ανατροφοδότηση σχετικά με την εμφάνισή του, ώστε να συμπεράνει πόσο υγιής φαίνεται ή να αποκτήσει αυτοπεποίθηση, αλλά και σχετικά με την ορθότητα της στάσης και της κίνησής του όταν αθλείται, όπως για παράδειγμα στα γυμναστήρια. Ως εκ τούτου, ο έξυπνος καθρέφτης μπορεί να αποδειχθεί χρήσιμος συμπληρώνοντας τις εφαρμογές του απλού καθρέφτη στους τομείς της υγείας και της ευεξίας. Δυστυχώς, όμως, μέχρι και σήμερα δεν έχει αναπτυχθεί ικανός αριθμός εφαρμογών για τον έξυπνο καθρέφτη που αφορούν στην υγεία.


 
\section{Σκοπός - Συνεισφορά της Διπλωματικής Εργασίας}
\label{section:contribution}

Η παρούσα διπλωματική έχει δύο πρωταρχικούς στόχους, την ανάπτυξη ενός modular λειτουργικού συστήματος το οποίο θα επιτρέπει στους χρήστες να αλληλεπιδράσουν με τον έξυπνο καθρέφτη και την ανάπτυξη μιας εφαρμογής που αφορά στον έλεγχο της ορθότητας μιας άσκησης ενός αθλούμενου. Ο έλεγχος του καθρέφτη από τον χρήστη επιτυγχάνεται μέσω φωνητικών εντολών, ενώ ο έλεγχος της ορθότητας μέσω αποθηκευμένων ασκήσεων, καθορισμένες από ειδικούς, με τις οποίες συγκρίνεται η κίνηση του χρήστη.

Αρχικά, υλοποιήθηκε ένα ελαφρύ γραφικό περιβάλλον για την εμφάνιση πληροφοριών γύρω από το είδωλό του χρήστη. Στη συνέχεια, το λογισμικό εμπλουτίστηκε με λογική για ανάγνωση φωνητικών εντολών και αναγνώριση πρόθεσης με σκοπό την εκτέλεση των κατάλληλων πράξεων, όπως ανανέωση μιας ρύθμισης ή προβολή μιας νέας πληροφορίας. Τέλος, δόθηκε ιδιαίτερη μνεία στην επεκτασιμότητα του συστήματος, δηλαδή στην δυνατότητα εύκολης εγκατάστασης εξωτερικών εφαρμογών από τον χρήστη οι οποίες επαυξάνουν τις λειτουργίες του καθρέφτη.

Όσον αφορά τον δεύτερο στόχο της διπλωματικής, έγινε υλοποίηση μιας εφαρμογής για την εκτίμηση της ορθότητας μιας άσκησης. Η εφαρμογή αντιλαμβάνεται τον χρήστη μέσω κάμερας και κάνει εκτίμηση της πόζας του κατά τη διάρκεια εκτέλεσης της άσκησης. Τα δεδομένα της πόζας συγκρίνονται με ένα "ground truth", το οποίο ορίζεται από κάποιον ειδικό γυμναστή ή φυσιοθεραπευτή, και δίνεται ανατροφοδότηση στον χρήστη σε πραγματικό χρόνο σχετικά με την ακρίβεια εκτέλεσης της άσκησης. Επιπλέον, παρέχεται η δυνατότητα στον χρήστη να αποθηκεύσει την άσκησή του με σκοπό να την αναπαράξει ξανά στο μέλλον.
\section{Διάρθρωση της Αναφοράς}
\label{section:layout}

Η διάρθρωση της παρούσας διπλωματικής εργασίας είναι η εξής:

\begin{itemize}
	\item{\textbf{Κεφάλαιο \ref{chapter:theory}:} Αναλύεται το θεωρητικό υπόβαθρο. 
	}
	\item{\textbf{Κεφάλαιο \ref{chapter:tools}:} Παρουσιάζονται τα διάφορα εργαλεία που χρησιμοποιήθηκαν στις υλοποιήσεις.
	}
	\item{\textbf{Κεφάλαιο \ref{chapter:implementation}:} Παρουσιάζεται η υλοποίηση του λογισμικού του καθρέφτη που αναπτύχθηκε.
	}
	\item{\textbf{Κεφάλαιο \ref{chapter:conclusions}:} Παρουσιάζονται τα τελικά συμπεράσματα και προτείνονται θέματα για μελλοντική μελέτη, αλλαγές και επεκτάσεις.
	}

\end{itemize}


