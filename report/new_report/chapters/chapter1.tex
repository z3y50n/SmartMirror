\chapter{Εισαγωγή}
\label{chapter:intro}

\setlength{\parskip}{1em}

Αν και υπήρχε ως ιδέα εδώ και περίπου 50 χρόνια, το διαδίκτυο των πραγμάτων, και ως όρος αλλά και ως προς τη χρήση του, ήρθε στο επίκεντρο του ενδιαφέροντος τα τελευταία 10 χρόνια και καθημερινά γίνεται ολοένα και πιο διαδεδομένο. Πλέον μάλιστα ο συνολικός αριθμός συνδεδεμένων συσκευών στο διαδίκτυο είναι μεγαλύτερος από αυτόν του πληθυσμού της Γης\footnote{\url{https://www.statista.com/statistics/471264/iot-number-of-connected-devices-worldwide/}}.

Το IoT βρίσκει εφαρμογή στην ανάπτυξη έξυπνων υποδομών και αυτοματοποίησης διαδικασιών. Από την δημιουργία ενός "Έξυπνου Σπιτιού" μέχρι και σε κάτι τόσο ουσιώδες όπως την καλύτερη παρακολούθηση ασθενών σε νοσοκομεία και άρα την πιο σωστή περίθαλψή τους. Άλλα παραδείγματα εφαρμογής είναι οι "Έξυπνες πόλεις" (διαχείρηση κυκλοφορίας στους δρόμους, διαχείριση απορριμάτων, διανομή νερού κ.α.), τα αυτοκινούμενα οχήματα, οι αυτοματισμοί στη γεωργία, και γενικότερα στη βιομηχανία.

Η ολοένα και μεγαλύτερη διάδοση του IoT, συνεπάγεται και την αξιοποίησή του από μεγαλύτερο κοινό, στο οποίο ανήκουν και άτομα τα οποία μπορεί να γνωρίζουν σε βάθος ένα συγκεκριμένο αντικείμενο στο οποίο βρίσκει εφαρμογή το IoT, όμως δεν έχουν επαρκείς, ή και καθόλου γνώσεις προγραμματισμού (οι λεγόμενοι citizen developers). Κρίνεται σκόπιμη λοιπόν η ανάπτυξη μεθόδων που θα μετατρέπουν την δημιουργία ενός συστήμος IoT σε διαδικασία πιο φιλική προς τα άτομα αυτά. Ταυτόχρονα, καθώς η αυτοματοποίηση διαδικασιών φαίνεται να χρησιμεύει σε όλο και περισσότερους τομείς, έχει αρχίσει να εμφανίζεται η \textit{ανάπτυξη χαμηλού-κώδικα} (\textit{low-code development}), η οποία αποσκοπεί σε όσο το δυνατό λιγότερη χρήση κώδικα για την παραγωγή λογισμικού. 

Εδώ βρίσκει άμεση εφαρμογή η Μοντελοστρεφής Μηχανική, η οποία προσφέρει γρήγορη και πιο αυτοματοποιημένη ανάπτυξη λογισμικού. Στο πλαίσιο αυτής, οι γλώσσες συγκεκριμένου πεδίου προσφέρουν ένα πιο αφαιρετικό επίπεδο για την παραγωγή λογισμικού, επομένως αποτελούν ένα πολύ σημαντικό εργαλείο για προγραμματιστές και μη, είτε για να απλοποιηθεί η διαδικασία παραγωγής για τους πρώτους, είτε για να καλυφθεί το κενό προγραμματιστικών γνώσεων για τους δεύτερους.

\section{Περιγραφή του Προβλήματος}
\label{section:problem_description}

Οι περισσότερες εφαρμογές που έχουν αναπτυχθεί πάνω στον έξυπνο καθρέφτη αυτή τη στιγμή αφορούν κυρίως στην ενημέρωση (news and social media feeds), στον συγχρονισμό ηλεκτρονικού ημερολογίου για λήψη υπενθυμίσεων και στην προβολή του καιρού αποτελώντας αρωγό σε καθημερινές δραστηριότητες. Όμως, οι απαιτήσεις χρηστών των παραπάνω εφαρμογών ικανοποιούνται ευκολότερα και αποτελεσματικότερα από άλλες συσκευές, όπως smartphone, καθιστώντας τον έξυπνο καθρέφτη μη ελκυστικό εμπορικό προϊόν. 

Βασική αιτία των παραπάνω αποτελεί το εγγενές πρόβλημα αλληλεπίδρασης με τον καθρέφτη, δηλαδή η έλλειψη αποδοτικών συσκευών εισόδου. Η επιλογή που φαντάζει λογικότερη ως είσοδο αποτελεί η φωνή του χρήστη δεδομένου ότι η επαφή του με τον καθρέφτη γίνεται από απόσταση. Όμως, στην ανίχνευση φωνής ελλοχεύουν δύο κύρια προβλήματα. Αρχικά, η αβεβαιότητα των συνθηκών του περιβάλλοντος στο οποίο τοποθετείται ο καθρέφτης δυσχεραίνει την ικανότητα του συστήματος να ανιχνεύει καθαρά τον ήχο. Επιπλέον, οι σημερινές τεχνολογίες αναγνώρισης φωνής και κατανόησης πρόθεσης του χρήστη επιδέχονται βελτίωση.

Επίσης, έως και σήμερα δεν έχει υλοποιηθεί κάποιο λειτουργικό σύστημα, το οποίο να χρησιμοποιείται καθολικά, κανονικοποιώντας έτσι τον τρόπο ανάπτυξης εφαρμογών πάνω στον έξυπνο καθρέφτη. Επομένως, οι λειτουργίες του κάθε συστήματος καθορίζονται από τον εκάστοτε κατασκευαστή μειώνοντας με αυτόν τον τρόπο την επεκτασιμότητα του έξυπνου καθρέφτη.

Παρόλα αυτά, ο έξυπνος καθρέφτης αποτελεί πρόσφορο έδαφος για την ανάπτυξη εφαρμογών που επαυξάνουν την βασική του λειτουργία, την προβολή του ειδώλου. Συνεχώς ο άνθρωπος αξιοποιεί έναν καθρέφτη προκειμένου να πάρει ανατροφοδότηση σχετικά με την εμφάνισή του, ώστε να συμπεράνει πόσο υγιής φαίνεται ή να αποκτήσει αυτοπεποίθηση, αλλά και σχετικά με την ορθότητα της στάσης και της κίνησής του όταν αθλείται, όπως για παράδειγμα στα γυμναστήρια. Ως εκ τούτου, ο έξυπνος καθρέφτης μπορεί να αποδειχθεί χρήσιμος συμπληρώνοντας τις εφαρμογές του απλού καθρέφτη στους τομείς της υγείας και της ευεξίας. Δυστυχώς, όμως, μέχρι και σήμερα δεν έχει αναπτυχθεί ικανός αριθμός εφαρμογών για τον έξυπνο καθρέφτη που αφορούν στην υγεία.


 
\section{Σκοπός - Συνεισφορά της Διπλωματικής Εργασίας}
\label{section:contribution}

Η παρούσα διπλωματική έχει δύο πρωταρχικούς στόχους, την ανάπτυξη ενός modular λειτουργικού συστήματος το οποίο θα επιτρέπει στους χρήστες να αλληλεπιδράσουν με τον έξυπνο καθρέφτη και την ανάπτυξη μιας εφαρμογής που αφορά στον έλεγχο της ορθότητας μιας άσκησης ενός αθλούμενου. Ο έλεγχος του καθρέφτη από τον χρήστη επιτυγχάνεται μέσω φωνητικών εντολών, ενώ ο έλεγχος της ορθότητας μέσω αποθηκευμένων ασκήσεων, καθορισμένες από ειδικούς, με τις οποίες συγκρίνεται η κίνηση του χρήστη.

Αρχικά, υλοποιήθηκε ένα ελαφρύ γραφικό περιβάλλον για την εμφάνιση πληροφοριών γύρω από το είδωλό του χρήστη. Στη συνέχεια, το λογισμικό εμπλουτίστηκε με λογική για ανάγνωση φωνητικών εντολών και αναγνώριση πρόθεσης με σκοπό την εκτέλεση των κατάλληλων πράξεων, όπως ανανέωση μιας ρύθμισης ή προβολή μιας νέας πληροφορίας. Τέλος, δόθηκε ιδιαίτερη μνεία στην επεκτασιμότητα του συστήματος, δηλαδή στην δυνατότητα εύκολης εγκατάστασης εξωτερικών εφαρμογών από τον χρήστη οι οποίες επαυξάνουν τις λειτουργίες του καθρέφτη.

Όσον αφορά τον δεύτερο στόχο της διπλωματικής, έγινε υλοποίηση μιας εφαρμογής για την εκτίμηση της ορθότητας μιας άσκησης. Η εφαρμογή αντιλαμβάνεται τον χρήστη μέσω κάμερας και κάνει εκτίμηση της πόζας του κατά τη διάρκεια εκτέλεσης της άσκησης. Τα δεδομένα της πόζας συγκρίνονται με ένα "ground truth", το οποίο ορίζεται από κάποιον ειδικό γυμναστή ή φυσιοθεραπευτή, και δίνεται ανατροφοδότηση στον χρήστη σε πραγματικό χρόνο σχετικά με την ακρίβεια εκτέλεσης της άσκησης. Επιπλέον, παρέχεται η δυνατότητα στον χρήστη να αποθηκεύσει την άσκησή του με σκοπό να την αναπαράξει ξανά στο μέλλον.
\section{Διάρθρωση της Αναφοράς}
\label{section:layout}

Η διάρθρωση της παρούσας διπλωματικής εργασίας είναι η εξής:

\begin{itemize}
	\item{\textbf{Κεφάλαιο \ref{chapter:theory}:} Αναλύεται το θεωρητικό υπόβαθρο. 
	}
	\item{\textbf{Κεφάλαιο \ref{chapter:tools}:} Παρουσιάζονται τα διάφορα εργαλεία που χρησιμοποιήθηκαν στις υλοποιήσεις.
	}
	\item{\textbf{Κεφάλαιο \ref{chapter:implementation}:} Παρουσιάζεται η υλοποίηση του λογισμικού του καθρέφτη που αναπτύχθηκε.
	}
	\item{\textbf{Κεφάλαιο \ref{chapter:conclusions}:} Παρουσιάζονται τα τελικά συμπεράσματα και προτείνονται θέματα για μελλοντική μελέτη, αλλαγές και επεκτάσεις.
	}

\end{itemize}


