\begin{center}
  \centering

  \vspace{0.5cm}
  \centering
  \textbf{\Large{Περίληψη}}
  \phantomsection
  \addcontentsline{toc}{section}{Περίληψη}

  \vspace{1cm}

\end{center}

Το διαδίκτυο των πραγμάτων (\textit{Internet of Things ή IoT}) είναι ένας κλάδος που εξελίσσεται ραγδαία ειδικά τα τελευταία χρόνια. Υπάρχει η δυνατότητα ανάπτυξης όλο και περισσότερων εφαρμογών, χρήσιμες για πολλούς ανθρώπους, είτε έχουν να κάνουν με απλές λειτουργίες σε συστήματα αυτοματισμού, είτε με μεγαλύτερης κλίμακας εφαρμογές στη βιομηχανία. Επομένως, όλο και περισσότερος κόσμος επιθυμεί να ασχοληθεί με το αντικείμενο αυτό.

Η διαδικασία υλοποίησης ενός IoT συστήματος περιλαμβάνει την ανάπτυξη κώδικα για τον έλεγχο των συσκευών. Μάλιστα, στις περισσότερες περιπτώσεις η γρήγορη απόκριση είναι υψίστης σημασίας, επομένως απαιτείται η ανάπτυξη χαμηλού επιπέδου κώδικα, καθώς και η χρήση λειτουργικών συστημάτων πραγματικού χρόνου (\textit{Real Time Operating System ή RTOS}). Επίσης, λόγω της μεγάλης ετερογένειας IoT συσκευών που υπάρχουν στην αγορά, κρίνεται αναγκαία η κατανόηση των δυνατοτήτων που η εκάστοτε συσκευή μπορεί να προσφέρει, ώστε να γίνεται η κατάλληλη επιλογή τους, προσαρμοσμένη στις ανάγκες του συστήματος προς υλοποίηση.

Οι ενέργειες αυτές είναι λογικό να φαίνονται περίπλοκες σε κάποιους χρήστες, ειδικότερα στα άτομα που είναι τεχνολογικά ακατάρτιστα, δεν έχουν δηλαδή τις απαραίτητες προγραμματιστικές γνώσεις, αλλά παρόλα αυτά επιθυμούν να κατασκευάσουν ένα IoT σύστημα π.χ. για προσωπική τους χρήση. Αυτό έχει ως αποτέλεσμα μεγάλη μερίδα κόσμου που θέλει να ασχοληθεί με το IoT να αποθαρρύνεται.

Η μοντελοστρεφής μηχανική (\textit{Model Driven Engineering ή MDE}), έρχεται να δώσει λύση στα προβλήματα που μπορεί να αντιμετωπίσουν όσοι/ες θέλουν να ασχοληθούν με το IoT, αλλά και γενικότερα να απλοποιήσει τη διαδικασία παραγωγής λογισμικού, καθώς μπορεί να παρέχει την ανάπτυξη IoT συστημάτων σε ένα πιο αφαιρετικό επίπεδο, το οποίο είναι πιο φιλικό προς τον απλό χρήστη.

Μέσω της παρούσας διπλωματικής εργασίας, δίνεται η δυνατότητα σε κάποιον/α να περιγράψει, με χρήση μοντέλων, IoT συσκευές, μέσω δύο γλωσσών συγκεκριμένου πεδίου (\textit{Domain Specific Language ή DSL}) που αναπτύχθηκαν, για την περιγραφή των συσκευών και των μεταξύ τους συνδέσεων. Από τα μοντέλα πραγματοποιείται ένας Model-to-Text μετασχηματισμός για την αυτόματη παραγωγή λογισμικού, για μια πληθώρα IoT συσκευών, προσαρμοσμένη στα χαρακτηριστικά που επιθυμεί ο χρήστης να έχει. Το λογισμικό ελέγχου των IoT συσκευών που παράγεται υλοποιεί την λήψη μετρήσεων από αισθητήρες και την αποστολή τους σε κάποιον μεσολαβητή (\textit{broker}), αλλά και τον έλεγχο ενεργοποιητών μέσω του broker. Επίσης αποτελείται από χαμηλού επιπέδου κώδικα, καθώς έχει σχεδιαστεί σύμφωνα με τις απαιτήσεις ενός λειτουργικού συστήματος πραγματικού χρόνου, το RIOT. Τέλος, πραγματοποιείται και ένας Model-to-Model μετασχηματισμός για την παραγωγή διαγραμμάτων τα οποία βοηθούν στην οπτικοποίηση και άρα καλύτερη αντίληψη από τον χρήστη για τη συνδεσμολογία και ενδοεπικοινωνία του συστήματός του.