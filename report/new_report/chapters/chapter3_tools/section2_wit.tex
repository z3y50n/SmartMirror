\section{Wit.ai}
\label{sec:wit}

\begin{figure}[h]
    \centering
    \includegraphics[scale=0.6]{images/chapter3/wit_logo.png}
    \caption{Wit.ai}
\end{figure}
Το Wit\footnote{\href{https://wit.ai/}{https://wit.ai/}} είναι ένα framework ανοιχτού κώδικα που επιτρέπει του χρήστες να αλληλεπιδρούν με την εφαρμογή μέσως φωνής και βασίζεται στη χρήση ισχυρού NLP (Natural Language Processing). Η λειτουργία του Wit μπορεί να χωριστεί σε 2 βασικές κατηγορίες, την \textbf{ταξινόμηση πρόθεσης} (intent classification) και την \textbf{εξαγωγή οντοτήτων} (entity extraction).

Το Wit επιτρέπει μια εφαρμογή να καταλάβει τους χρήστες. Αρχικά μπορεί ο προγραμματιστής μέσω της πλατφόρμας να δώσει κάποιες \textbf{εκφράσεις} (utterances) ως παραδείγματα στις οποίες ορίζει την πρόθεση και τις οντότητες που περιέχουν προκειμένου να εκπαιδευτεί το μοντέλο. Για παράδειγμα μπορούμε να δώσουμε ως έκφραση την παρακάτω πρόταση:

\enquote{What's the temperature?}

\noindent και να ορίσουμε ως πρόθεση \texttt{temperature\_get}. Με αυτόν τον τρόπο μαθαίνουμε στο Wit ότι με αυτήν την έκφραση ο χρήστης δηλώνει την πρόθεσή του να μάθει την θερμοκρασία. Όταν επομένως ρωτήσει ο χρήστης το Wit για την θερμοκρασία (είτε με την παραπάνω έκφραση είτε με μια παρόμοια όπως \enquote{I would like to know the temperature}) το Wit θα προσπαθήσει να προβλέψει την πρόθεση του χρήστη δίνοντας και μία μετρική της σιγουριάς (confidence) από 0 έως 1. Όσα περισσότερα παραδείγματα δώσουμε τόσο μεγαλύτερη γίνεται η σιγουριά και η ικανότητα του Wit για την πρόβλεψη της πρόθεσης.

Έστω ότι δίνεται ως είσοδος στο Wit η έκφραση \enquote{set the temperature to 70 degrees}. Η απάντηση που θα δοθεί θα έχει την εξής μορφή:

\begin{lstlisting}[]
{
    "text": "set the temperature to 70 degrees",
    "intents": [
        {
            "id": "226127658493500",
            "name": "temperature_get",
            "confidence": 0.9953
        }
    ],
    "entities": [],
    "traits": []
}
\end{lstlisting}

Η απάντηση όμως δεν είναι ικανοποιητική αφού η πρόθεση του χρήστη είναι να αλλάξει η θερμοκρασία και όχι να πάρει την θερμοκρασία κάτι που είναι λογικό αφού το Wit καταλαβαίνει μόνο μία πρόθεση. Μέσω της πλατφόρμας όμως είναι εύκολο να αλλάξουμε την πρόθεση μιας ήδη δοσμένης έκφρασης και να ξαναεκπαιδευτεί το μοντέλο θέτοντας αυτή τη φορά την πρόθεση \texttt{temperature\_set}. Δίνεται επίσης η επιλογή να οριστεί ένα κατώφλι σιγουριάς έτσι ώστε να μην επιστρέφονται προθέσεις κάτω από αυτό το όριο.

Στο παραπάνω παράδειγμα, θέλουμε να αποσπάσουμε μαζί με την πρόθεση και το νούμερο της επιθυμητής θερμοκρασίας. Κάτι τέτοιο είναι εύκολο μέσω της πλατφόρμας όπου μπορούμε να ορίσουμε το \texttt{70 degrees} να είναι μια οντότητα (πχ \texttt{temperature\_degrees}). Έτσι στην απάντηση που θα λάβει η εφαρμογή από το Wit θα μπορεί να εξάγει και τον αριθμό των βαθμών έτσι ώστε να προβεί στην κατάλληλη πράξη για να θέσει την θερμοκρασία στην επιθυμητή τιμή.

Η επιλογή του Wit έγινε ανάμεσα σε αρκετά αντίστοιχα εργαλεία ανοιχτού κώδικα που υπάρχουν διαθέσιμα όπως το spaCy\footnote{\href{https://spacy.io/}{https://spacy.io/}} ή το Rasa\footnote{\href{https://rasa.com/open-source/}{https://rasa.com/open-source/}}. Ο βασικός λόγος επιλογής του Wit είναι η ευκολία εκπαίδευσης και προσαρμογής του μέσω της πλατφόρμας. Επίσης, η ενσωμάτωση στην εφαρμογή ήταν πολύ εύκολη με τη χρήση της βιβλιοθήκης\footnote{\href{https://github.com/wit-ai/pywit}{https://github.com/wit-ai/pywit}} που παρέχει το Wit για Python, το documentation ήταν αρκετά κατατοπιστικό και η δομή των δεδομένων στην απάντηση του Wit το έκαναν εύκολο στην χρήση. Τέλος, είναι δυνατό κανείς να κατεβάσει τα δεδομένα ενός ήδη εκπαιδευμένου μοντέλου, και να δημιουργήσει ένα με βάση αυτά όποτε και να μπορεί στη συνέχει να το επεκτείνει με δικές του εντολές και να το προσαρμόσει στις ανάγκες των δικών του εφαρμογών, βοηθώντας κατά αυτόν τον τρόπο στην επεκτασιμότητα της διεπαφής του καθρέφτη.