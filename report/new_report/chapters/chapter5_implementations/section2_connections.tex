\section{Ορισμός μετα-μοντέλου Συνδέσεων}
\label{sec:metamodel_connections}

Το μετα-μοντέλο αυτό περιέχει χαρακτηριστικά που μπορεί να έχει το σύστημα που επιθυμεί ο χρήστης να κατασκευάσει όσον αφορά στη συνδεσμολογία μεταξύ των συσκευών αλλά και στη σύνδεσή τους σε έναν broker. Στο \autoref{fig:metamodel_connections} μπορούμε να δούμε μία απεικόνισή του.

\begin{figure}[!ht]
	\centering
	\includegraphics[width=1.0\textwidth]{./images/chapter5/metamodel_connections.png}
	\caption{Μετα-μοντέλο συνδέσεων}
	\label{fig:metamodel_connections}
\end{figure}

\subsection{SYSTEM}
\label{subsec:system}

\subsubsection*{Σύνοψη}

\noindent Το στοιχείο αυτό αναπαριστά ένα σύστημα, το οποίο αποτελείται από συνδέσεις μεταξύ συσκευών. Οι συσκευές αυτές μπορεί να είναι είτε μικροελεγκτές είτε περιφερειακά. 

\subsubsection*{Ιδιότητες και Συσχετίσεις}

\begin{table}[H]
	\begin{center}
		\caption{Συσχετίσεις του \textit{SYSTEM}.}
		\label{tab:system}
		\begin{tabular}{ | c | c | c| m{5.5cm} | }
			\hline
			\rowcolor{Gray}
			Όνομα & Τύπος & Πολλαπλότητα & Περιγραφή \\
			\hline
			INCLUDE & Composition-Σύνθεση & 0..* &  Τα ονόματα των συσκευών που απαρτίζουν το σύστημα \\
			\hline
			CONNECTION & Composition-Σύνθεση & 0..* &  Οι συνδέσεις μεταξύ των συσκευών \\
			\hline
		\end{tabular}
	\end{center}
\end{table}

\noindent Δεν περιλαμβάνει περαιτέρω ιδιότητες.

\subsubsection*{Περιορισμοί}

\noindent Δεν υπάρχουν περιορισμοί.

\subsection{INCLUDE}
\label{subsec:include}

\subsubsection*{Σύνοψη}

\noindent Το στοιχείο αυτό είναι το όνομα μιας συσκευής η οποία είναι μέρος του συστήματος, ώστε να ξέρουμε την ύπαρξή της.

\subsubsection*{Ιδιότητες και Συσχετίσεις}

\begin{table}[H]
	\begin{center}
		\caption{Ιδιότητες του \textit{INCLUDE}.}
		\label{tab:include}
		\begin{tabular}{ | c | c | c| m{5.5cm} | }
			\hline
			\rowcolor{Gray}
			Όνομα & Τύπος & Πολλαπλότητα & Περιγραφή \\
			\hline
			name & ID & 1..1 &  Το όνομα της συσκευής \\
			\hline
		\end{tabular}
	\end{center}
\end{table}

\noindent Δεν περιλαμβάνει περαιτέρω συσχετίσεις.

\subsubsection*{Περιορισμοί}

\noindent Το ID πρέπει να είναι το όνομα συσκευής για την οποία υπάρχει configuration file (.hwd), και άρα να υποστηρίζεται από την παρούσα εργασία.

\subsection{CONNECTION}
\label{subsec:connection}

\subsubsection*{Σύνοψη}

\noindent Το στοιχείο αυτό αναπαριστά τη σύνδεση ενός μικροελεγκτή με ένα περιφερειακό.

\subsubsection*{Ιδιότητες και Συσχετίσεις}

\begin{table}[H]
	\begin{center}
		\caption{Ιδιότητες του \textit{CONNECTION}.}
		\label{tab:connection1}
		\begin{tabular}{ | c | c | c| m{5.5cm} | }
			\hline
			\rowcolor{Gray}
			Όνομα & Τύπος & Πολλαπλότητα & Περιγραφή \\
			\hline
			name & ID & 1..1 &  Το όνομα που θα δοθεί στη σύνδεση \\
			\hline
		\end{tabular}
	\end{center}
\end{table}

\begin{table}[H]
	\begin{center}
		\caption{Συσχετίσεις του \textit{CONNECTION}.}
		\label{tab:connection2}
		\begin{tabular}{ | c | c | c| m{5.5cm} | }
			\hline
			\rowcolor{Gray}
			Όνομα & Τύπος & Πολλαπλότητα & Περιγραφή \\
			\hline
			PERIPHERAL & Composition-Σύνθεση & 1..1 &  Το περιφερειακό της συγκεκριμένης σύνδεσης \\
			\hline
			BOARD & Composition-Σύνθεση & 1..1 &  Ο μικροελεγκτής της συγκεκριμένης σύνδεσης \\
			\hline
			\scriptsize{POWER\_CONNECTION} & Composition-Σύνθεση & 0..* &  Οι ακροδέκτες που χρησιμοποιούνται για την τροφοδοσία \\
			\hline
			\footnotesize{HW\_CONNECTION} & Composition-Σύνθεση & 1..* &  Οι ακροδέκτες που χρησιμοποιούνται για τη σύνδεση μεταξύ των διεπαφών υλικού \\
			\hline
			\small{COM\_ENDPOINT} & Composition-Σύνθεση & 0..1 &  Τα χαρακτηριστικά σύνδεσης σε κάποιον broker \\
			\hline
		\end{tabular}
	\end{center}
\end{table}

\subsubsection*{Περιορισμοί}

\noindent Δεν υπάρχουν περιορισμοί.

\subsection{PERIPHERAL}
\label{subsec:peripheral_con}

\subsubsection*{Σύνοψη}

\noindent Το στοιχείο αυτό περιγράφει το περιφερειακό που χρησιμοποιείται σε μία σύνδεση.

\subsubsection*{Ιδιότητες και Συσχετίσεις}

\begin{table}[H]
	\begin{center}
		\caption{Ιδιότητες του \textit{PERIPHERAL}.}
		\label{tab:peripheral_con}
		\begin{tabular}{ | c | c | c| m{5.5cm} | }
			\hline
			\rowcolor{Gray}
			Όνομα & Τύπος & Πολλαπλότητα & Περιγραφή \\
			\hline
			device & ID & 1..1 & Το όνομα του περιφερειακού \\
			\hline
		    number & INT & 0..1 & Ο αριθμός του περιφερειακού σε περίπτωση που χρησιμοποιείται το ίδιο μοντέλο πολλαπλές φορές στο συγκεκριμένο σύστημα \\
			\hline
		\end{tabular}
	\end{center}
\end{table}

\noindent Δεν περιλαμβάνει περαιτέρω συσχετίσεις.

\subsubsection*{Περιορισμοί}

\noindent Δεν υπάρχουν περιορισμοί.

\subsection{BOARD}
\label{subsec:board_con}

\subsubsection*{Σύνοψη}

\noindent Το στοιχείο αυτό περιγράφει τον μικροελεγκτή που χρησιμοποιείται σε μία σύνδεση.

\subsubsection*{Ιδιότητες και Συσχετίσεις}

\begin{table}[H]
	\begin{center}
		\caption{Ιδιότητες του \textit{BOARD}.}
		\label{tab:board_con}
		\begin{tabular}{ | c | c | c| m{5.5cm} | }
			\hline
			\rowcolor{Gray}
			Όνομα & Τύπος & Πολλαπλότητα & Περιγραφή \\
			\hline
			device & ID & 1..1 & Το όνομα του μικροελεγκτή \\
			\hline
			number & INT & 0..1 & Ο αριθμός του μικροελεγκτή σε περίπτωση που χρησιμοποιείται το ίδιο μοντέλο πολλαπλές φορές στο συγκεκριμένο σύστημα \\
			\hline
		\end{tabular}
	\end{center}
\end{table}

\noindent Δεν περιλαμβάνει περαιτέρω συσχετίσεις.

\subsubsection*{Περιορισμοί}

\noindent Δεν υπάρχουν περιορισμοί.

\subsection{POWER\_CONNECTION}
\label{subsec:power_connection}

\subsubsection*{Σύνοψη}

\noindent Το στοιχείο αυτό περιγράφει τη σύνδεση (ακροδέκτες) τροφοδοσίας των περιφερειακών από τους μικροελεγκτές.

\subsubsection*{Ιδιότητες και Συσχετίσεις}

\begin{table}[H]
	\begin{center}
		\caption{Ιδιότητες του \textit{POWER\_CONNECTION}.}
		\label{tab:power_connection}
		\begin{tabular}{ | c | c | c| m{5.5cm} | }
			\hline
			\rowcolor{Gray}
			Όνομα & Τύπος & Πολλαπλότητα & Περιγραφή \\
			\hline
			board\_power & ID & 1..1 & Ο ακροδέκτης τροφοδοσίας του μικροελεγκτή \\
			\hline
			peripheral\_power & ID & 1..1 & Ο ακροδέκτης τροφοδοσίας του περιφερειακού \\
			\hline
		\end{tabular}
	\end{center}
\end{table}

\noindent Δεν περιλαμβάνει περαιτέρω συσχετίσεις.

\subsubsection*{Περιορισμοί}

\noindent Δεν υπάρχουν περιορισμοί.

\subsection{COM\_ENDPOINT}
\label{subsec:com_endpoint}

\subsubsection*{Σύνοψη}

\noindent Το στοιχείο αυτό περιγράφει τα στοιχεία της σύνδεσης μιας συσκευής σε έναν broker, καθώς και το topic στο οποίο κατά τη λειτουργία της θα κάνει είτε subscribe είτε publish.

\subsubsection*{Ιδιότητες και Συσχετίσεις}

\begin{table}[H]
	\begin{center}
		\caption{Ιδιότητες του \textit{COM\_ENDPOINT}.}
		\label{tab:com_endpoint1}
		\begin{tabular}{ | c | c | c| m{5.5cm} | }
			\hline
			\rowcolor{Gray}
			Όνομα & Τύπος & Πολλαπλότητα & Περιγραφή \\
			\hline
			topic & NEW\_ID & 1..1 & Το όνομα του topic στο οποίο θα στο οποίο θα γίνει publish ή subscribe (ανάλογα αν χρησιμοποιείται αισθητήρας ή ενεργοποιητής αντίστοιχα) \\
			\hline
			wifi\_ssid & NEW\_ID & 1..1 & Το όνομα του wifi δικτύου στο οποίο θα συνδεθεί ο μικροελεγκτής \\
			\hline
			wifi\_password & NEW\_ID & 1..1 & Ο κωδικός του wifi δικτύου στο οποίο θα συνδεθεί ο μικροελεγκτής \\
			\hline
			addr & ADDRESS\_ID & 1..1 & Η IPv6 διεύθυνση του broker με τον οποίο θα επικοινωνήσει ο μικροελεγκτής \\
			\hline
			port & INT & 1..1 & Η πύλη σύνδεσης του broker \\
			\hline
		\end{tabular}
	\end{center}
\end{table}

\begin{table}[H]
	\begin{center}
		\caption{Συσχετίσεις του \textit{COM\_ENDPOINT}.}
		\label{tab:com_endpoint2}
		\begin{tabular}{ | c | c | c| m{5.5cm} | }
			\hline
			\rowcolor{Gray}
			Όνομα & Τύπος & Πολλαπλότητα & Περιγραφή \\
			\hline
			MSG\_ENTRIES & Composition-Σύνθεση & 1..1 &  Τα είδη μηνυμάτων που θα διαμοιραστούν κατά τη συγκεκριμένη σύνδεση \\
			\hline
			FREQUENCY & Composition-Σύνθεση & 0..1 &  Η συχνότητα με την οποία θα γίνονται publish τα μηνύματα στον broker \\
			\hline
		\end{tabular}
	\end{center}
\end{table}

\subsubsection*{Περιορισμοί}

\noindent Τα topic, wifi\_ssid και wifi\_password μπορούν να πάρουν τιμές σαν ID, με επιπλέον επιλογή να περιλαμβάνουν και παύλες. Αυτό επιτυγχάνεται σύμφωνα με το NEW\_ID που είναι μια \textit{κανονική έκφραση} (\textit{regex}).

\noindent Το addr μπορεί να πάρει τιμές σαν μια διεύθυνση \textit{IPv6} (\textit{Internet Protocol version 6}). Αυτό επιτυγχάνεται σύμφωνα με το ADDRESS\_ID που είναι μια \textit{κανονική έκφραση} (\textit{regex}).

\subsection{MSG\_ENTRIES}
\label{subsec:msg_entries}

\subsubsection*{Σύνοψη}

\noindent Το στοιχείο αυτό περιγράφει τα είδη μηνυμάτων που θα διαμοιραστούν σε μια συγκεκριμένη σύνδεση.

\subsubsection*{Ιδιότητες και Συσχετίσεις}

\begin{table}[H]
	\begin{center}
		\caption{Ιδιότητες του \textit{MSG\_ENTRIES}.}
		\label{tab:msg_entries}
		\begin{tabular}{ | c | c | c| m{5.5cm} | }
			\hline
			\rowcolor{Gray}
			Όνομα & Τύπος & Πολλαπλότητα & Περιγραφή \\
			\hline
			msg\_entries & MSG\_TYPES (Enum) & 1..* & Τα είδη μηνυμάτων \\
			\hline
		\end{tabular}
	\end{center}
\end{table}

\noindent Δεν περιλαμβάνει περαιτέρω συσχετίσεις.

\subsubsection*{Περιορισμοί}

\noindent Επιλογές των ειδών μηνυμάτων που μπορούν να δηλωθούν:

\begin{itemize}
	\item Distance
	\item Temperature
	\item Humidity
	\item Gas
	\item Pressure
	\item Env
	\item Acceleration
	\item Motor\_Controller
	\item Leds\_Controller
	\item Servo\_Controller
\end{itemize}

\subsection{HW\_CONNECTION}
\label{subsec:hw_connection}

\subsubsection*{Σύνοψη}

\noindent Το στοιχείο αυτό είναα η abstract κλάση για την περιγραφή των συνδέσεων των διεπαφών υλικών των συσκευών.

\subsubsection*{Ιδιότητες και Συσχετίσεις}

\noindent Δεν περιλαμβάνει περαιτέρω ιδιότητες και συσχετίσεις.

\subsubsection*{Περιορισμοί}

\noindent Δεν υπάρχουν περιορισμοί.

\subsection{GPIO}
\label{subsec:gpio_con}

\subsubsection*{Σύνοψη}

\noindent Το στοιχείο αυτό περιγράφει τη σύνδεση δύο GPIO διεπαφών.

\subsubsection*{Ιδιότητες και Συσχετίσεις}

\begin{table}[H]
	\begin{center}
		\caption{Ιδιότητες του \textit{GPIO}.}
		\label{tab:gpio_con1}
		\begin{tabular}{ | c | c | c| m{5.5cm} | }
			\hline
			\rowcolor{Gray}
			Όνομα & Τύπος & Πολλαπλότητα & Περιγραφή \\
			\hline
			type & STRING & 1..1 & Το είδος διεπαφής (στην προκειμένη περίπτωση gpio) \\
			\hline
			board\_int & ID & 1..1 & Η διεπαφή του μικροελεγκτή \\
			\hline
			peripheral\_int & ID & 1..1 & Η διεπαφή του περιφερειακού \\
			\hline
		\end{tabular}
	\end{center}
\end{table}

\begin{table}[H]
	\begin{center}
		\caption{Συσχετίσεις του \textit{GPIO}.}
		\label{tab:gpio_con2}
		\begin{tabular}{ | c | c | c| m{5.5cm} | }
			\hline
			\rowcolor{Gray}
			Όνομα & Τύπος & Πολλαπλότητα & Περιγραφή \\
			\hline
			\footnotesize{HW\_CONNECTION} & SuperType-Επέκταση & - &  Το στοιχείο GPIO επεκτείνει το στοιχείο HW\_CONNECTION \\
			\hline
		\end{tabular}
	\end{center}
\end{table}

\subsubsection*{Περιορισμοί}

\noindent To type θα πρέπει να έχει την τιμή "gpio", αλλιώς θα εμφανιστεί σφάλμα.

\subsection{I2C}
\label{subsec:i2c_con}

\subsubsection*{Σύνοψη}

\noindent Το στοιχείο αυτό περιγράφει τη σύνδεση μέσω πρωτοκόλλου I2C.

\subsubsection*{Ιδιότητες και Συσχετίσεις}

\begin{table}[H]
	\begin{center}
		\caption{Ιδιότητες του \textit{I2C}.}
		\label{tab:i2c_con1}
		\begin{tabular}{ | c | c | c| m{5.5cm} | }
			\hline
			\rowcolor{Gray}
			Όνομα & Τύπος & Πολλαπλότητα & Περιγραφή \\
			\hline
			type & STRING & 1..1 & Το είδος διεπαφής (στην προκειμένη περίπτωση i2c) \\
			\hline
			board\_int & list[ID] & 1..1 & Οι διεπαφές του μικροελεγκτή \\
			\hline
			peripheral\_int & list[ID] & 1..1 & Οι διεπαφές του περιφερειακού \\
			\hline
			slave\_addr & ID & 1..1 & Η διεύθυνση της διεπαφής που λειτουργεί ως slave \\
			\hline
		\end{tabular}
	\end{center}
\end{table}

\begin{table}[H]
	\begin{center}
		\caption{Συσχετίσεις του \textit{I2C}.}
		\label{tab:i2c_con2}
		\begin{tabular}{ | c | c | c| m{5.5cm} | }
			\hline
			\rowcolor{Gray}
			Όνομα & Τύπος & Πολλαπλότητα & Περιγραφή \\
			\hline
			\footnotesize{HW\_CONNECTION} & SuperType-Επέκταση & - &  Το στοιχείο I2C επεκτείνει το στοιχείο HW\_CONNECTION \\
			\hline
		\end{tabular}
	\end{center}
\end{table}

\subsubsection*{Περιορισμοί}

\noindent To type θα πρέπει να έχει την τιμή "i2c", αλλιώς θα εμφανιστεί σφάλμα.

\subsection{SPI}
\label{subsec:spi_con}

\subsubsection*{Σύνοψη}

\noindent Το στοιχείο αυτό περιγράφει τη σύνδεση μέσω πρωτοκόλλου SPI.

\subsubsection*{Ιδιότητες και Συσχετίσεις}

\begin{table}[H]
	\begin{center}
		\caption{Ιδιότητες του \textit{SPI}.}
		\label{tab:spi_con1}
		\begin{tabular}{ | c | c | c| m{5.5cm} | }
			\hline
			\rowcolor{Gray}
			Όνομα & Τύπος & Πολλαπλότητα & Περιγραφή \\
			\hline
			type & STRING & 1..1 & Το είδος διεπαφής (στην προκειμένη περίπτωση spi) \\
			\hline
			board\_int & list[ID] & 1..1 & Οι διεπαφές του μικροελεγκτή \\
			\hline
			peripheral\_int & list[ID] & 1..1 & Οι διεπαφές του περιφερειακού \\
			\hline
		\end{tabular}
	\end{center}
\end{table}

\begin{table}[H]
	\begin{center}
		\caption{Συσχετίσεις του \textit{SPI}.}
		\label{tab:spi_con2}
		\begin{tabular}{ | c | c | c| m{5.5cm} | }
			\hline
			\rowcolor{Gray}
			Όνομα & Τύπος & Πολλαπλότητα & Περιγραφή \\
			\hline
			\footnotesize{HW\_CONNECTION} & SuperType-Επέκταση & - &  Το στοιχείο SPI επεκτείνει το στοιχείο HW\_CONNECTION \\
			\hline
		\end{tabular}
	\end{center}
\end{table}

\subsubsection*{Περιορισμοί}

\noindent To type θα πρέπει να έχει την τιμή "spi", αλλιώς θα εμφανιστεί σφάλμα.

\subsection{UART}
\label{subsec:uart_con}

\subsubsection*{Σύνοψη}

\noindent Το στοιχείο αυτό περιγράφει τη σύνδεση μέσω πρωτοκόλλου UART.

\subsubsection*{Ιδιότητες και Συσχετίσεις}

\begin{table}[H]
	\begin{center}
		\caption{Ιδιότητες του \textit{UART}.}
		\label{tab:uart_con1}
		\begin{tabular}{ | c | c | c| m{5.5cm} | }
			\hline
			\rowcolor{Gray}
			Όνομα & Τύπος & Πολλαπλότητα & Περιγραφή \\
			\hline
			type & STRING & 1..1 & Το είδος διεπαφής (στην προκειμένη περίπτωση uart) \\
			\hline
			board\_int & list[ID] & 1..1 & Οι διεπαφές του μικροελεγκτή \\
			\hline
			peripheral\_int & list[ID] & 1..1 & Οι διεπαφές του περιφερειακού \\
			\hline
			baudrate & ID & 1..1 & Το baudrate που χρησιμοποιείται \\
			\hline
		\end{tabular}
	\end{center}
\end{table}

\begin{table}[H]
	\begin{center}
		\caption{Συσχετίσεις του \textit{UART}.}
		\label{tab:uart_con2}
		\begin{tabular}{ | c | c | c| m{5.5cm} | }
			\hline
			\rowcolor{Gray}
			Όνομα & Τύπος & Πολλαπλότητα & Περιγραφή \\
			\hline
			\footnotesize{HW\_CONNECTION} & SuperType-Επέκταση & - &  Το στοιχείο UART επεκτείνει το στοιχείο HW\_CONNECTION \\
			\hline
		\end{tabular}
	\end{center}
\end{table}

\subsubsection*{Περιορισμοί}

\noindent To type θα πρέπει να έχει την τιμή "uart", αλλιώς θα εμφανιστεί σφάλμα.

\subsection{FREQUENCY}
\label{subsec:frequency}

\subsubsection*{Σύνοψη}

\noindent Το στοιχείο αυτό περιγράφει τη συχνότητα με την οποία θα γίνονται publish τα μηνύματα στον broker.

\subsubsection*{Ιδιότητες και Συσχετίσεις}

\begin{table}[H]
	\begin{center}
		\caption{Ιδιότητες του \textit{FREQUENCY}.}
		\label{tab:frequency}
		\begin{tabular}{ | c | c | c| m{5.5cm} | }
			\hline
			\rowcolor{Gray}
			Όνομα & Τύπος & Πολλαπλότητα & Περιγραφή \\
			\hline
			val & INT & 1..1 & Η τιμή της συχνότητας \\
			\hline
			val & FREQ\_UNIT & 1..1 & Η μονάδα μέτρησης της συχνότητας \\
			\hline
		\end{tabular}
	\end{center}
\end{table}

\noindent Δεν περιλαμβάνει περαιτέρω συσχετίσεις.

\subsubsection*{Περιορισμοί}

\noindent Επιλογές των υποστηριζόμενων μονάδων μέτρησης για το max\_freq:

\begin{itemize}
	\item hz
	\item khz
	\item ghz
	\item mhz
\end{itemize}