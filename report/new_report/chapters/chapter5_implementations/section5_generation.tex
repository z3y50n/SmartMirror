\section{Παραγωγή κώδικα}
\label{sec:generation}

Σε αυτήν την ενότητα θα εξεταστεί η διαδικασία παραγωγής αρχείων κώδικα από την παρούσα εργασία. Τα αρχεία που παράγονται είναι δύο, ένα αρχείο σε γλώσσα C και ένα αρχείο Makefile, από τα οποία θα παραχθεί ένα εκτελέσιμο για να φορτωθεί σε κάποιον μικροελεγκτή. Τα αρχεία αυτά παράγονται μέσω κάποιων \textit{πρότυπων αρχείων}\footnote{\url{https://github.com/robotics-4-all/2020_riot_mde_thanos_manolis/tree/master/riot_mde/templates}} (\textit{templates}) που δημιουργήθηκαν στα πλαίσια της διπλωματικής.

Όπως αναφέρθηκε στο \autoref{sec:transformation} αρχικά παράγονται τα μοντέλα για κάθε μια από της συσκευές που χρησιμοποιούνται στην εκάστοτε υλοποίηση, αλλά και για τις μεταξύ τους συνδέσεις. Η πληροφορία που αντλείται από τα μοντέλα, δίνεται με τη μορφή παραμέτρων στα πρότυπα αρχεία. Με αυτόν τον τρόπο, από τα πρότυπα αρχεία παράγονται τα τελικά αρχεία προς εκτέλεση. Παρακάτω αναλύονται κάποιες βασικές συναρτήσεις και λειτουργίες που υλοποιούνται από τα πρότυπα αρχεία.

\subsection{Αρχείο κώδικα C}
\label{subsec:c_code}

\subsubsection{Επικοινωνία με broker}

\noindent Για την επικοινωνία του μικροελεγκτή με κάποιον broker υλοποιήθηκαν οι παρακάτω συναρτήσεις.

\noindent\begin{minipage}{\textwidth}
\noindent $\bullet$ con( addr , port )

\begin{table}[H]
	\centering
	\begin{tabular}{|c|c|}
		\hline
		\rowcolor{Gray}
		\multicolumn{2}{|c|}{\textbf{Περιγραφή}} \\ 
		\hline
		\multicolumn{2}{|c|}{Σύνδεση στον broker} \\ 
		\hline
		\rowcolor{Gray}
		\multicolumn{2}{|c|}{\textbf{Ορίσματα}}  \\
		\hline
		\rowcolor{Gray} 
		Όρισμα & Επεξήγηση \\
		\hline
		addr & IPv6 διεύθυνση του broker \\ 
		\hline
		port & Πύλη επικοινωνίας MQTT-SN \\
		\hline
	\end{tabular}
	\label{tab:con}
\end{table}
\end{minipage}

\noindent\begin{minipage}{\textwidth}
\noindent $\bullet$ discon( )

\begin{table}[H]
	\centering
	\begin{tabular}{|c|c|}
		\hline
		\rowcolor{Gray}
		\multicolumn{2}{|c|}{\textbf{Περιγραφή}} \\ 
		\hline
		\multicolumn{2}{|c|}{Αποσύνδεση από τον broker} \\ 
		\hline
		\rowcolor{Gray}
		\multicolumn{2}{|c|}{\textbf{Ορίσματα}}  \\
		\hline
		\multicolumn{2}{|c|}{\textbf{-}}  \\
		\hline
	\end{tabular}
	\label{tab:discon}
\end{table}
\end{minipage}

\noindent\begin{minipage}{\textwidth}
	\noindent $\bullet$ pub(topic, data, qos)
	
	\begin{table}[H]
		\centering
		\begin{tabular}{|c|c|}
			\hline
			\rowcolor{Gray}
			\multicolumn{2}{|c|}{\textbf{Περιγραφή}} \\ 
			\hline
			\multicolumn{2}{|c|}{Publish κάποιου μηνύματος σε συγκεκριμένο topic} \\ 
			\hline
			\rowcolor{Gray}
			\multicolumn{2}{|c|}{\textbf{Ορίσματα}}  \\
			\hline
			\rowcolor{Gray} 
			Όρισμα & Επεξήγηση \\
			\hline
			topic & Topic στο οποίο θα γίνει το publish \\ 
			\hline
			data & Το μήνυμα που πρόκειται να γίνει publish \\
			\hline
			qos & Ποιότητα υπηρεσιών (Quality of Service) \\
			\hline
		\end{tabular}
		\label{tab:pub}
	\end{table}
\end{minipage}

\noindent\begin{minipage}{\textwidth}
	\noindent $\bullet$ sub(topic, qos, func)
	
	\begin{table}[H]
		\centering
		\begin{tabular}{|c|M{13cm}|}
			\hline
			\rowcolor{Gray}
			\multicolumn{2}{|c|}{\textbf{Περιγραφή}} \\ 
			\hline
			\multicolumn{2}{|c|}{Publish κάποιου μηνύματος σε συγκεκριμένο topic} \\ 
			\hline
			\rowcolor{Gray}
			\multicolumn{2}{|c|}{\textbf{Ορίσματα}}  \\
			\hline
			\rowcolor{Gray} 
			Όρισμα & Επεξήγηση \\
			\hline
			topic & Topic στο οποίο θα "ακούει" \\ 
			\hline
			qos & Ποιότητα υπηρεσιών (Quality of Service) \\
			\hline
			func & Η συνάρτηση που θα είναι υπεύθυνη για την εκτέλεση λειτουργιών σε περίπτωση κάποιου publish \\
			\hline
		\end{tabular}
		\label{tab:sub}
	\end{table}
\end{minipage}

\subsubsection{Sensor}

\noindent $\bullet$ send\_<sensor\_name>( )

\noindent Στην περίπτωση που χρησιμοποιείται αισθητήρας, η συνάρτησή αρχικά κάνει εκκίνηση του αισθητήρα, πραγματοποιεί μία μέτρηση, και τέλος την κάνει publish στο αντίστοιχο topic στον broker. Όπου sensor\_name θα εμφανιστεί το όνομα του αισθητήρα.

\subsubsection{Actuator}

\noindent $\bullet$ receive\_<actuator\_name>( )

\noindent Η συνάρτηση αυτή είναι υπεύθυνη ώστε κάθε φορά που γίνεται publish στο αντίστοιχο topic, να καλεί την επόμενη συνάρτηση ( \textit{<actuator\_name>\_on\_pub( )} ) ώστε να εκτελείται η διαδικασία για τους ενεργοποιητές. Όπου actuator\_name θα εμφανιστεί το όνομα του ενεργοποιητή.

\noindent $\bullet$ <actuator\_name>\_on\_pub( )

Στην περίπτωση όπου γίνει κάποιο publish στο αντίστοιχο topic, η συνάρτηση αυτή είναι υπεύθυνη ώστε να αποθηκεύσει το μήνυμα, να κάνει εκκίνηση του ενεργοποιητή και τέλος να δράσει ανάλογα με το είδος του ενεργοποιητή. Όπου actuator\_name θα εμφανιστεί το όνομα του ενεργοποιητή.

\subsection{Αρχείο Makefile}
\label{subsec:makefile}

Το Makefile είναι αρχείο απαραίτητο για να γίνει compile του εκτελέσιμου κώδικα C. Στο RIOT υπάρχουν κάποια υλοποιημένα modules τα οποία μπορούν να χρησιμοποιηθούν στις εφαρμογές που αναπτύσσονται με το λειτουργικό αυτό. Στο Makefile γίνονται τα includes των απαραίτητων modules. Στην παρούσα εργασία χρησιμοποιούνται modules που είναι υπεύθυνα για λειτουργίες δικτύωσης, χρονοδιακοπτών, καθώς και τα modules όλων των περιφερειακών που χρησιμοποιούνται στην εκάστοτε υλοποίηση.

Το όνομα της υλοποίησης, τα ονόματα των περιφερειακών και του μικροελεγκτή καθώς και τα στοιχεία του wifi στο οποίο θα συνδεθεί, δίνονται ως παράμετροι στο πρότυπο αρχείο Makefile.