\section{textX}
\label{sec:textx}

Η textX \cite{bib:textx} είναι μία μετα-γλώσσα και ένα εργαλείο για τη δημιουργία γλωσσών συγκεκριμένου τομέα (DSL) σε Python. Δηλαδή, αυτή η μετα-γλώσσα παρέχει τη γραμματική για των ορισμό νέων γλωσσών. Για κάθε γραμματική η textX δημιουργεί έναν αναλυτή και ένα μέτα-μοντέλο. Το μετά μοντέλο εμπεριέχει όλες τις πληροφορίες σχετικά με την γλώσσα και από την γραμματική παράγεται ένα σύνολο από κλάσεις στην γλώσσα Python. Ο αναλυτής θα προσπελάσει τα προγράμματα/μοντέλα που έχουν γραφεί σε αυτήν την νέα γλώσσα και θα δημιουργήσει έναν γράφο αντικειμένων σε Python (δηλαδή ένα μοντέλο) που θα συμμορφώνεται στο μετα-μοντέλο. 

Το εργαλείο textX παρέχει υποστήριξη για αναφορά και εντοπισμό σφαλμάτων, καθώς και οπτικοποίηση για τα μετα-μοντέλα και τα μοντέλα που παράγονται. Είναι ανοιχτού κώδικα και είναι διαθέσιμο στο GitHub \footnotetext{\url{https://github.com/textX/textX}}.

Στην παρούσα εργασία χρησιμοποιήθηκε για τον ορισμό των DSL που περιγράφουν τα μοντέλα των συσκευών και των μεταξύ τους δυνδέσεων.

\begin{figure}[!ht]
  \centering
  \includegraphics[width=0.4\textwidth]{./images/chapter4/textx.png}
  \caption[textX]{textX\footnotemark}
  \label{fig:textx}
\end{figure}

\footnotetext{\url{http://textx.github.io/textX/stable/}}