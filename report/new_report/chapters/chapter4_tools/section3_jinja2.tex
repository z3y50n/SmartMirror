\section{Jinja}
\label{sec:jinja}

Το \textit{Jinja}\footnote{\url{https://jinja.palletsprojects.com/en/3.0.x/}} είναι ένα εργαλείο για παραγωγή αρχείων κειμένων μέσω προτύπων. Μέσω της χρήσης παραμέτρων, υπάρχει η δυνατότητα δυναμικής τροποποίησης των αρχείων αυτών. Αρχικά δημιουργήθηκε για την παραγωγή αρχείων HTML, αλλά μπορεί να χρησιμοποιηθεί για την παραγωγή αρχείων κειμένου σε οποιαδήποτε γλώσσα προγραμμτισμού. Το Jinja είναι βιβλιοθήκη της Python, και άρα η διαδικασία παραγωγής αρχείων από τα πρότυπα, θα πρέπει να οριστεί σε ένα αρχείο κειμένου γραμμένο σε Python. 

Στην παρούσα εργασία χρησιμοποιήθηκε για την παραγωγή αρχείων Makefile και κώδικα σε γλώσσα C ώστε να δημιουργηθεί ένα πρόγραμμα που θα τρέχει στο λειτουργικό RIOT.

\begin{figure}[!ht]
  \centering
  \includegraphics[width=0.5\textwidth]{./images/chapter4/jinja.png}
  \caption{Jinja}
  \label{fig:jinja}
\end{figure}