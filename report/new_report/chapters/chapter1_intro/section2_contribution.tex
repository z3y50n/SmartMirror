\section{Σκοπός - Συνεισφορά της Διπλωματικής Εργασίας}
\label{section:contribution}

Η παρούσα διπλωματική έχει δύο πρωταρχικούς στόχους, την ανάπτυξη ενός modular λειτουργικού συστήματος το οποίο θα επιτρέπει στους χρήστες να αλληλεπιδράσουν με τον έξυπνο καθρέφτη και την ανάπτυξη μιας εφαρμογής που αφορά στον έλεγχο της ορθότητας μιας άσκησης ενός αθλούμενου. Ο έλεγχος του καθρέφτη από τον χρήστη επιτυγχάνεται μέσω φωνητικών εντολών, ενώ ο έλεγχος της ορθότητας μέσω αποθηκευμένων ασκήσεων, καθορισμένες από ειδικούς, με τις οποίες συγκρίνεται η κίνηση του χρήστη.

Αρχικά, υλοποιήθηκε ένα ελαφρύ γραφικό περιβάλλον για την εμφάνιση πληροφοριών γύρω από το είδωλό του χρήστη. Στη συνέχεια, το λογισμικό εμπλουτίστηκε με λογική για ανάγνωση φωνητικών εντολών και αναγνώριση πρόθεσης με σκοπό την εκτέλεση των κατάλληλων πράξεων, όπως ανανέωση μιας ρύθμισης ή προβολή μιας νέας πληροφορίας. Τέλος, δόθηκε ιδιαίτερη μνεία στην επεκτασιμότητα του συστήματος, δηλαδή στην δυνατότητα εύκολης εγκατάστασης εξωτερικών εφαρμογών από τον χρήστη οι οποίες επαυξάνουν τις λειτουργίες του καθρέφτη.

Όσον αφορά τον δεύτερο στόχο της διπλωματικής, έγινε υλοποίηση μιας εφαρμογής για την εκτίμηση της ορθότητας μιας άσκησης. Η εφαρμογή αντιλαμβάνεται τον χρήστη μέσω κάμερας και κάνει εκτίμηση της πόζας του κατά τη διάρκεια εκτέλεσης της άσκησης. Τα δεδομένα της πόζας συγκρίνονται με ένα "ground truth", το οποίο ορίζεται από κάποιον ειδικό γυμναστή ή φυσιοθεραπευτή, και δίνεται ανατροφοδότηση στον χρήστη σε πραγματικό χρόνο σχετικά με την ακρίβεια εκτέλεσης της άσκησης. Επιπλέον, παρέχεται η δυνατότητα στον χρήστη να αποθηκεύσει την άσκησή του με σκοπό να την αναπαράξει ξανά στο μέλλον.


%Η παρούσα διπλωματική έχει ως στόχο την ανάπτυξη μιας μηχανής λογισμικού μοντελοστρεφούς λογικής, με την οποία οι χρήστες θα μπορούν να μοντελοποιούν συσκευές καθώς και την διασύνδεσή τους. Οι συσκευές αυτές θα μπορούν να είναι είτε μικροελεγκτές, είτε περιφερειακά (αισθητήρες και ενεργοποιητές), και όλα μαζί θα συνδέονται κατάλληλα για να συνθέσουν ένα σύστημα.
%
%Αρχικά υλοποιήθηκαν δύο DSL για την περιγραφή των συσκευών και των μεταξύ τους συνδέσεων. Στην μία περιγράφονται τα χαρακτηριστικά των συσκευών (μνήμη, μονάδα επεξεργασίας, δικτύωση, ακροδέκτες κ.α.) και στην άλλη οι μεταξύ τους συνδέσεις (συνδέσεις ακροδεκτών, πρωτόκολλα επικοινωνίας που χρησιμοποιούνται κ.α.). Μέσω αυτών, δημιουργούνται τα κατάλληλα μοντέλα για τις συσκευές και συνδέσεις.
%
%Από τα μοντέλα, πραγματοποιούνται δύο μετασχηματισμοί, ένας Model-to-Text (M2T) και ένας Model-to-Model (M2M). Ο M2M έχει ως αποτέλεσμα την παραγωγή διαγραμμάτων, τα οποία βοηθούν στην οπτικοποίηση της συνδεσμολογίας και ενδοεπικοινωνίας του συστήματος. Μέσω του M2T, παράγονται αυτόματα τμήματα λογισμικού που θα υλοποιούν κάποιες βασικές λειτουργίες (λήψη μετρήσεων από αισθητήρες, έλεγχος ενεργοποιητών κ.α.). Ο παραγόμενος κώδικας θα αφορά συσκευές που υποστηρίζονται από το λειτουργικό σύστημα πραγματικού χρόνου RIOT.