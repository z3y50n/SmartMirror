\section{Περιγραφή του Προβλήματος}
\label{section:problem_description}

Οι περισσότερες εφαρμογές που έχουν αναπτυχθεί πάνω στον έξυπνο καθρέφτη αυτή τη στιγμή αφορούν κυρίως στην ενημέρωση (news and social media feeds), στον συγχρονισμό ηλεκτρονικού ημερολογίου για λήψη υπενθυμίσεων και στην προβολή του καιρού αποτελώντας αρωγό σε καθημερινές δραστηριότητες. Όμως, οι απαιτήσεις χρηστών των παραπάνω εφαρμογών ικανοποιούνται ευκολότερα και αποτελεσματικότερα από άλλες συσκευές, όπως smartphone, καθιστώντας τον έξυπνο καθρέφτη μη ελκυστικό εμπορικό προϊόν. 

Βασική αιτία των παραπάνω αποτελεί το εγγενές πρόβλημα αλληλεπίδρασης με τον καθρέφτη, δηλαδή η έλλειψη αποδοτικών συσκευών εισόδου. Η επιλογή που φαντάζει λογικότερη ως είσοδο αποτελεί η φωνή του χρήστη δεδομένου ότι η επαφή του με τον καθρέφτη γίνεται από απόσταση. Όμως, στην ανίχνευση φωνής ελλοχεύουν δύο κύρια προβλήματα. Αρχικά, η αβεβαιότητα των συνθηκών του περιβάλλοντος στο οποίο τοποθετείται ο καθρέφτης δυσχεραίνει την ικανότητα του συστήματος να ανιχνεύει καθαρά τον ήχο. Επιπλέον, οι σημερινές τεχνολογίες αναγνώρισης φωνής και κατανόησης πρόθεσης του χρήστη επιδέχονται βελτίωση.

Επίσης, έως και σήμερα δεν έχει υλοποιηθεί κάποιο λειτουργικό σύστημα, το οποίο να χρησιμοποιείται καθολικά, κανονικοποιώντας έτσι τον τρόπο ανάπτυξης εφαρμογών πάνω στον έξυπνο καθρέφτη. Επομένως, οι λειτουργίες του κάθε συστήματος καθορίζονται από τον εκάστοτε κατασκευαστή μειώνοντας με αυτόν τον τρόπο την επεκτασιμότητα του έξυπνου καθρέφτη.

Παρόλα αυτά, ο έξυπνος καθρέφτης αποτελεί πρόσφορο έδαφος για την ανάπτυξη εφαρμογών που επαυξάνουν την βασική του λειτουργία, την προβολή του ειδώλου. Συνεχώς ο άνθρωπος αξιοποιεί έναν καθρέφτη προκειμένου να πάρει ανατροφοδότηση σχετικά με την εμφάνισή του, ώστε να συμπεράνει πόσο υγιής φαίνεται ή να αποκτήσει αυτοπεποίθηση, αλλά και σχετικά με την ορθότητα της στάσης και της κίνησής του όταν αθλείται, όπως για παράδειγμα στα γυμναστήρια. Ως εκ τούτου, ο έξυπνος καθρέφτης μπορεί να αποδειχθεί χρήσιμος συμπληρώνοντας τις εφαρμογές του απλού καθρέφτη στους τομείς της υγείας και της ευεξίας. Δυστυχώς, όμως, μέχρι και σήμερα δεν έχει αναπτυχθεί ικανός αριθμός εφαρμογών για τον έξυπνο καθρέφτη που αφορούν στην υγεία.


 